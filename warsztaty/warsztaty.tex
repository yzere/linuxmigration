\documentclass[10pt,a4paper]{article}

\usepackage[T1]{fontenc}
\usepackage[polish]{babel}
\usepackage[utf8]{inputenc}
\usepackage{graphicx}
\usepackage[export]{adjustbox}

\graphicspath{ {./media/} }

\usepackage[margin=1.2in]{geometry}

\usepackage{hyperref}
\hypersetup{
    colorlinks=true,
    linkcolor=black,
    filecolor=magenta,      
    urlcolor=blue,
}

\usepackage{amsfonts}
\usepackage{titlesec}

\titleformat{\chapter}[hang] 
{\normalfont\huge\bfseries}{\chaptertitlename\ \thechapter:}{1em}{} 
\titleformat*{\section}{\huge}
\begin{document}

\title{\Huge Warsztaty wprowadzające do systemu operacyjnego Linux}
\author{Maciej Tracz \\\\Technikum Mechatroniczne nr 1 w Warszawie}
\date{Rok 2020}
\maketitle

\textbf{Dokumentacja wymaga pełnego przeczytania przed realizacją warsztatów.}

\tableofcontents

\newpage

\section{Podstawy}

\textbf{Czas trwania wrasztatów:} 60-120 minut (zależy od poziomu grupy oraz ilości wykonanych zadań)\\\\
\textbf{Poziom trudności:} Początkujący i osoby mało-zaawansowane\\\\

\textbf{Wymagania sprzętowe:} \\\\ Osoby uczestniczące muszą mieć możliwość
\begin{enumerate}
\item tworzenia maszyn wirtualnych
\item pobrania ISO docelowego systemu operacyjnego\\
\end{enumerate}

\textbf{Początek warsztatów:}
\begin{enumerate}
\item Osoba prowadząca powinna przedstawić wszystkim uczesnikom kopię dokumentacji migracji na systemy Linux. Pomoże to rozumieć przekazywane treści merytoryczne.
\item Prowadzący wyjaśnia budowę systemu operacyjnego i pyta uczestników o zrozumienie zagadnienia.
\item Uczestnicy dostają zadanie przedstawienia najważniejszych różnic między znanymi im systemami operacyjnymi. Jeśli osoby te nie posiadają takiej wiedzy prowadzący powinien zaproponować przejrzenie rozdziału dokumentacji skierowanego temu tematowi.
\item Prowadzący kończy wstęp i przechodzi do daleszego etapu warsztatów.
\end{enumerate}

\textbf{Materiały przydatne prowadzącemu} (wycinki z dokumentacji): \\

[...] Ale czym właściwie jest system operacyjny?\\

Jest to oprogramowanie systemowe, które zapewnia kontrolę i komunikację nad podzespołami urządzenia, zasobami systemowymi oraz umożliwia korzystanie z serwisów tworzących środowisko robocze dla urzytkownika.

Składa się on z:
\begin{itemize}
\item Jądra systemowego - Służy on do wykonywania i kontrolowania zadań przydzielonych systemowi. Zarząda wszystkimi zasobami maszyny i komunikacją wszystkich elementów jednostki.
\item Powłoki - zapewnia komunikację między użytkownikiem a systemem operacyjnym

\begin{itemize}
\item Tekstowe - tak zwane 'terminale'. Pozwalają za pomocą komend zarządzać zasobami i wykonywać operacje
\item Graficzne(GUI) - interfejsy pozwalające użytkownikowi zobaczyć zasoby w bardziej przystępnej formie. Wspierają operacje myszką i ograniczają potrzebę znania komend do minimum.
\end{itemize}

\item Systemu plików - pozwala zapewnić dostęp i stworzyć strukturę plików dla urzytkownika. Dostarcza zabezpieczenia prywatności oraz szyfrowanie jeśli odpowiednio dobrane i skonfigurowane.
\item Aplikacje wbudowane - zestaw aplikacji zawartych w obrazie instalowanego systemu. To one dostarczają możliwości Out-Of-The-Box, pozwalające używać komputera już przy pierwszym logowaniu. Ten zestaw zależy zupełnie od twórców.
\end{itemize}
	
Ważnym jest tu zauważenie, że w przeciwieństwie do Windowsa i MacOS, które są jednolitymi systemami, Linux jest jądrem. Oznacza to w praktyce, że referując do Systemu operacyjnego linux mówimy o jego dystrybucjach, o czym więcej w późniejszych rozdziałach. [...]

\section{Przygotowanie}

Miejsce prowadzenia warsztatów: Pracownia komputerowa lub online \\\\


\section{Ćwiczenia}

\section{Dodatkowe materiały}

\end{document}