\documentclass[10pt,a4paper]{article}

\usepackage[T1]{fontenc}
\usepackage[polish]{babel}
\usepackage[utf8]{inputenc}
\usepackage{graphicx}
\usepackage[export]{adjustbox}

\graphicspath{ {./media/} }

\usepackage[margin=1.2in]{geometry}

\usepackage{hyperref}
\hypersetup{
    colorlinks=true,
    linkcolor=black,
    filecolor=magenta,      
    urlcolor=blue,
}

\usepackage{amsfonts}
\usepackage{titlesec}

\titleformat{\chapter}[hang] 
{\normalfont\huge\bfseries}{\chaptertitlename\ \thechapter:}{1em}{} 
\titleformat*{\section}{\huge}
\begin{document}

\title{\Huge Warsztaty wprowadzające do obsługi systemu operacyjnego Linux}
\author{Maciej Tracz \\\\Technikum Mechatroniczne nr 1 w Warszawie}
\date{Rok 2020}
\maketitle

\textbf{Dokumentacja wymaga pełnego przeczytania przed realizacją warsztatów.}

\tableofcontents

\newpage

\section{Podstawy}

\textbf{Czas trwania wrasztatów:} 60-120 minut (zależy od poziomu grupy oraz ilości wykonanych zadań)\\\\
\textbf{Poziom trudności:} Początkujący i osoby mało-zaawansowane\\\\

\textbf{Wymagania sprzętowe:} \\\\ Osoby uczestniczące muszą mieć możliwość
\begin{enumerate}
\item tworzenia maszyn wirtualnych
\item pobrania ISO docelowego systemu operacyjnego\\
\end{enumerate}

\textbf{Początek warsztatów:}
\begin{enumerate}
\item Osoba prowadząca powinna przedstawić wszystkim uczesnikom kopię dokumentacji migracji na systemy Linux. Pomoże to rozumieć przekazywane treści merytoryczne.
\item Prowadzący wyjaśnia budowę systemu operacyjnego i pyta uczestników o zrozumienie zagadnienia.
\item Uczestnicy dostają zadanie przedstawienia najważniejszych różnic między znanymi im systemami operacyjnymi. Jeśli osoby te nie posiadają takiej wiedzy prowadzący powinien zaproponować przejrzenie rozdziału dokumentacji skierowanego temu tematowi.
\item Prowadzący kończy wstęp i przechodzi do daleszego etapu warsztatów.
\end{enumerate}

\textbf{Materiały przydatne prowadzącemu} (wycinki z dokumentacji): \\

[...] Ale czym właściwie jest system operacyjny?\\

Jest to oprogramowanie systemowe, które zapewnia kontrolę i komunikację nad podzespołami urządzenia, zasobami systemowymi oraz umożliwia korzystanie z serwisów tworzących środowisko robocze dla urzytkownika.

Składa się on z:
\begin{itemize}
\item Jądra systemowego - Służy on do wykonywania i kontrolowania zadań przydzielonych systemowi. Zarząda wszystkimi zasobami maszyny i komunikacją wszystkich elementów jednostki.
\item Powłoki - zapewnia komunikację między użytkownikiem a systemem operacyjnym

\begin{itemize}
\item Tekstowe - tak zwane 'terminale'. Pozwalają za pomocą komend zarządzać zasobami i wykonywać operacje
\item Graficzne(GUI) - interfejsy pozwalające użytkownikowi zobaczyć zasoby w bardziej przystępnej formie. Wspierają operacje myszką i ograniczają potrzebę znania komend do minimum.
\end{itemize}

\item Systemu plików - pozwala zapewnić dostęp i stworzyć strukturę plików dla urzytkownika. Dostarcza zabezpieczenia prywatności oraz szyfrowanie jeśli odpowiednio dobrane i skonfigurowane.
\item Aplikacje wbudowane - zestaw aplikacji zawartych w obrazie instalowanego systemu. To one dostarczają możliwości Out-Of-The-Box, pozwalające używać komputera już przy pierwszym logowaniu. Ten zestaw zależy zupełnie od twórców.
\end{itemize}
	
Ważnym jest tu zauważenie, że w przeciwieństwie do Windowsa i MacOS, które są jednolitymi systemami, Linux jest jądrem. Oznacza to w praktyce, że referując do Systemu operacyjnego linux mówimy o jego dystrybucjach, o czym więcej w późniejszych rozdziałach. [...]

\section{Przygotowanie}

Miejsce prowadzenia warsztatów: Pracownia komputerowa lub online \\\\
Każdy uczestnik musi mieć dostęp do komputera z możliwością wirtualizacji.\\
Każdy uczestnik musi mieć zainstalowany program do wirtualizacji. Polecane:
\begin{itemize}
\item VirtualBox
\item VMWare Workstation Player
\item Hyper-V (Na Windows Pro oraz Enterprise)\\\\
\end{itemize}

Przygotowanie maszyn na warsztaty(w pracowni wykonuje się to przed zajęciami):
\begin{enumerate}
\item Prowadzący upewnia się, że każdy uczesnik posiada zainstalowane oprogramowanie do wirtualizacji.
\item Jeśli któryś z uczestników ma problemy z taką instalacją, prowadzący stara się pomóc takiej osobie aby mogła brać pełny udział w warsztatach.
\item Prowadzący wybiera ISO na którym będą prowadzone zajęcia. Zaleca się stosowanie popularnych i prostych dystrybucji, aby uniknąć wszelkich problemów.
\begin{itemize}
\item Ubuntu 20.04 lub nowsze
\item Linux Mint Cinnamon w najnowszej wersji
\item Debian 10 lub nowszy (przy wykluczeniu zadań obejmujących interfejs graficzny)
\item Systemu bazujące na archu oraz inne różniące się składnią wymagają własnej korekcji zadań.
\end{itemize}
\item \textbf{Ważne!} Osoby uczestniczące w warsztatach  online muszą pobrać wybrane przez prowadzącego ISO przed zajęciami z względu na różnice w prędkości łącza między lokacjami.
\item Jeżeli warsztaty odbywają się w pracowni to w razie możliwości moża utworzyć dysk wirtualny z zainstalowanym systemem i rozdystrybuować go pomiędzy wszystkie używane maszyny. Zaoszczędzi to czas zarówno przed jak i podczas zajęć.
\end{enumerate}
\newpage

\section{Ćwiczenia}

Ćwiczenia można zacząć gdy wszystkie osoby uczestniczące będą miały dostęp do zainstalowanego systemu i będą gotowe do rozpoczęcia pracy.

\subsection{Zadania wstępne}
Na początku pozwalamy uczestnikom samemu poeksperymentować z systemem. Da im to czas przyzwyczaić się do wyglądu oraz rozbudzić względną ciekawość. Na tym etapie prowadzący może wykazać się własną intencją i zadać proste ćwiczenia konfiguracji pulpitu lub innych ustawień.\\\\
Gdy uczestnicy skończą swobodną interakcję z maszynami prowadzący może przedstawić im podstawy systemu:

\begin{enumerate}
\item Prowadzący przedstawia podstawowe komendy konsolowe potrzebne do wykonania zadań:
\begin{itemize}
\item \textbf{\underline{echo}} - pozwala wypisać tekst w konsoli
\item \textbf{\underline{man}} - otwiera instrukcję obsługi komendy
\item \textbf{\underline{ls}} - listuje wszystkie pliki i foldery w aktualnym katalogu
\item \textbf{\underline{cd}} - ustawia katalog roboczy
\item \textbf{\underline{cat}} - wypisuje zawartość pliku do konsoli
\item \textbf{\underline{touch}} - tworzy nowe pliki
\item \textbf{\underline{useradd}} - tworzy nowego użytkownika
\item \textbf{\underline{su}}- loguje się do nowego użytkownika
\item \textbf{\underline{passwd}} - umożliwia zmianę hasła użytkownika. Wymaga komendy \textbf{sudo} 
\item \textbf{\underline{sudo}} - pozwala wykonywać komendy administratora. Tak zwany super user.\\\\
\end{itemize}


\item Po przestawieniu komend prowadzący prosi uczestników o wykonanie poszczególnych zadań. Uczestnicy mają korzystać z instrukcji obsługi \textbf{\underline{man [twoja\_ komenda]}} lub dopisując \textbf{-h} / \textbf{--help} aby uzyskać pomoc w zapisie.\\ \end{enumerate}
Zadania:
\begin{enumerate}
\item Wypisz "hello world" do terminalu.
\item Sprawdz możliwe operatory i opcje komendy \textbf{\underline{echo}}.
\item Wyświetl wszystkie pliki aktualnego katalogu.
\item Zmień katalog roboczy na /Desktop/.
\item Wyświetl pliki na pulpicie. Utwórz na nim folder Raport a w nim plik tekstowy Raport.txt przy pomocy interfejsu graficznego.
\item W pliku raport.txt zapisz wybrane przez ciebie zdanie.
\item W terminalu wejdz do utworzonego folderu i wypisz w terminalu zawartość pliku raport.txt.
\item Przez terminal swórz plik wyniki.txt
\item Dodaj użytkownika Janek używając praw administracyjnych.
\item Ustaw Jankowi hasło używając praw administracyjnych.
\item Zaloguj się na konto janka w terminalu.

\end{enumerate}

\subsection{Zadania dopełniające}

\begin{enumerate}
\item Prowadzący przedstawia podstawowe komendy konsolowe potrzebne do wykonania zadań:
\begin{itemize}
\item \textbf{\underline{Operatory}}
\begin{itemize}
\item \textgreater - wypisuje wynik komendy do pliku. Nadpisuje jeśli plik zawiera już dane.
\item \textgreater \textgreater - wypisuje wynik komendy do pliku. Dodaje treść do pliku jeśli już jakąś zawiera.
\item \& - pozwala ominąć czas wykonywania komendy i przejść od razu do terminalu gdy komenda działa w tle.
\item \& \& - uruchamia komendy po sobie, gdy każda ukończy swoje działanie. Wymaga poprawnego działanie każdej komendy.
\item \$ - deklaruje zmienne środowiskowe.
\item \textbar - pozwala zastosować wynik pierwszej komendy jako dane wejściowe dla drugiej.
\end{itemize}
\item \textbf{\underline{mkdir}} - tworzy katalogi
\item \textbf{\underline{mv}} - przenosi pliki lub katalogi.
\item \textbf{\underline{cp}} - kopiuje pliki lub katalogi.
\item \textbf{\underline{rm}} - usuwa pliki lub katalogi.
\item \textbf{\underline{chmod}} - definiuje uprawnienia pliku.
\item \textbf{\underline{chown}} - definiuje prawa własnościowe pliku.
\item \textbf{\underline{find}} - wyszukuje plików i katalogów o konkretnej nazwie w katalogu roboczym i jego podkatalogach.
\item \textbf{\underline{grep}} - wyszukuje konkretnego tekstu w pliku lub danych wejściowych z poprzedniej komendy.
\item \textbf{\underline{vi}} - konsolowy edytor plików. Najważniejsze informacje o nim to:
\begin{itemize}
\item Kliknij \textbf{I} aby wejść w tryb wprowadzania danych
\item Po pliku poruszaj się strzałkami.
\item Po zakończeniu wprowadzania danych kliknij guzik \textbf{ESC} aby wyjść z trybu wprowadzania.
\item Kliknij guzik \textbf{:} aby zacząć wykonywać operacje na pliku.
\begin{itemize}
\item w - zapisuje dane
\item wq - zapisuje dane i wychodzi z pliku
\item q! - usuwa wszystkie zmiany
\end{itemize}
\end{itemize}

\end{itemize}

\item Po przestawieniu komend prowadzący prosi uczestników o wykonanie poszczególnych zadań. Uczestnicy mają korzystać z instrukcji obsługi \textbf{\underline{man [twoja\_ komenda]}} lub dopisując \textbf{-h} / \textbf{--help} aby uzyskać pomoc w zapisie.\\ 
\end{enumerate}
Zadania tylko przy użyciu terminalu:
\begin{enumerate}
\item Wyczyść katalogi robocze. (Ta sama komenda co do ich istawiania)
\item Utwórz na koncie Janka folder Raporty.
\item W katalogu raporty utwórz dwa foldery Maj oraz Czerwiec.
\item W katalogu Maj utwórz pusty plik raport1.txt
\item Wypisz do pliku raport1.txt zawartość foldera /bin/.
\item Wypisz zawartość tego pliku.
\item Przenieś plik raport1.txt do folderu Czerwiec.
\item W folderze Maj swórz plik raport2.txt i wypisz do niego zawartość foldera /tmp/.
\item Skopiuj ten plik do katalogu Czerwiec.
\item Usuń z foldera Maj plik raport2.txt.
\item Wejdz na stronę github.com/yzere/bash-basics/ i przepisz zawartość pliku hello.sh do własnego pliku hello.sh stworzonego w folderze Maj.
\item Dodaj możliwość uruchamiania pliku przy pomocy argumentu +x.
\item Uruchom plik dzieki składni ./twój\_ plik
\item Znajdz wszystkie foldery /tmp.
\item Znajdz wszystkie foldery /tmp zawierające literę "a".
\item Do pliku hello.sh dodaj nową linię kodu z komendą echo i wybranym zdaniem.
\item Zapisz i wyjdz z pliku.
\item Uruchom go ponownie.
\end{enumerate}


\section{Dodatkowe materiały}

Ćwiczenia tu przedstawione można poszerzyć o tworzenie skryptów wykonujących poprzednie ćwiczenia automatyczni.\\ \par  Na bazie dokumenatcji można poprowadzić instalację oraz konfigurację kontenerów wraz z uruchomieniem serwera multimedialnego. Jest to jednak proces zaawansowany i może wyniknąć wieloma błędami gdy uczestnicy nie przyswoili jeszcze tak dobrze podstaw Linuxa.\\ \par Ćwiczenia można poszerzyć o wykorzytanie publicznych repozytoriów na github.com i naukę dodawania oraz instalowania pakietów. Polecane projekty to:
\begin{itemize}
\item Emulacja pakietu Microsoft Office na Linuxie.
\item Wykorzystanie projektu Sherlock do wyszukiwania kont o konkretnej nazwie.
\item Uruchamianie programów w kontenerach.
\item Instalacja serwera Minecraft Bukkit lub Vanila.
\item Skorzystanie z zasobów strony tryhackme.com do poznania zabezpieczeń i urzycia narzędzi na systemach Linux.\\\\
\end{itemize}


\end{document}